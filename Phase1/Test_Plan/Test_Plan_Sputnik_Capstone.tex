\documentclass[]{article}
\usepackage{lmodern}
\usepackage{amssymb,amsmath}
\usepackage{ifxetex,ifluatex}
\usepackage{fixltx2e} % provides \textsubscript
\ifnum 0\ifxetex 1\fi\ifluatex 1\fi=0 % if pdftex
  \usepackage[T1]{fontenc}
  \usepackage[utf8]{inputenc}
\else % if luatex or xelatex
  \ifxetex
    \usepackage{mathspec}
  \else
    \usepackage{fontspec}
  \fi
  \defaultfontfeatures{Ligatures=TeX,Scale=MatchLowercase}
\fi
% use upquote if available, for straight quotes in verbatim environments
\IfFileExists{upquote.sty}{\usepackage{upquote}}{}
% use microtype if available
\IfFileExists{microtype.sty}{%
\usepackage{microtype}
\UseMicrotypeSet[protrusion]{basicmath} % disable protrusion for tt fonts
}{}
\usepackage{hyperref}
\hypersetup{unicode=true,
            pdfborder={0 0 0},
            breaklinks=true}
\urlstyle{same}  % don't use monospace font for urls
\usepackage{longtable,booktabs}
\usepackage{graphicx,grffile}
\makeatletter
\def\maxwidth{\ifdim\Gin@nat@width>\linewidth\linewidth\else\Gin@nat@width\fi}
\def\maxheight{\ifdim\Gin@nat@height>\textheight\textheight\else\Gin@nat@height\fi}
\makeatother
% Scale images if necessary, so that they will not overflow the page
% margins by default, and it is still possible to overwrite the defaults
% using explicit options in \includegraphics[width, height, ...]{}
\setkeys{Gin}{width=\maxwidth,height=\maxheight,keepaspectratio}
\IfFileExists{parskip.sty}{%
\usepackage{parskip}
}{% else
\setlength{\parindent}{0pt}
\setlength{\parskip}{6pt plus 2pt minus 1pt}
}
\setlength{\emergencystretch}{3em}  % prevent overfull lines
\providecommand{\tightlist}{%
  \setlength{\itemsep}{0pt}\setlength{\parskip}{0pt}}
\setcounter{secnumdepth}{0}
% Redefines (sub)paragraphs to behave more like sections
\ifx\paragraph\undefined\else
\let\oldparagraph\paragraph
\renewcommand{\paragraph}[1]{\oldparagraph{#1}\mbox{}}
\fi
\ifx\subparagraph\undefined\else
\let\oldsubparagraph\subparagraph
\renewcommand{\subparagraph}[1]{\oldsubparagraph{#1}\mbox{}}
\fi

\date{}

\begin{document}

\section{Sputnik Capstone Test Plan}\label{sputnik-capstone-test-plan}

written by Shan Quinney, William Harrington

\textbf{Table of Contents} *
\href{https://github.com/wrh2/sputnik/wiki/Phase-1-Test-Plan\#revision-history}{Revision
History} *
\href{https://github.com/wrh2/sputnik/wiki/Phase-1-Test-Plan\#introduction}{Introduction}
*
\href{https://github.com/wrh2/sputnik/wiki/Phase-1-Test-Plan\#purpose}{Purpose}
*
\href{https://github.com/wrh2/sputnik/wiki/Phase-1-Test-Plan\#testing-procedure}{Testing
Procedure} *
\href{https://github.com/wrh2/sputnik/wiki/Phase-1-Test-Plan\#recording-of-results-witnessing-and-authorities}{Recording
of Results, witnessing, and authorities} *
\href{https://github.com/wrh2/sputnik/wiki/Phase-1-Test-Plan\#reference-documents}{Reference
Documents} *
\href{https://github.com/wrh2/sputnik/wiki/Phase-1-Test-Plan\#design-documentation}{Design
Documentation} *
\href{https://github.com/wrh2/sputnik/wiki/Phase-1-Test-Plan\#overview}{Overview}
*
\href{https://github.com/wrh2/sputnik/wiki/Phase-1-Test-Plan\#operational-description}{Operational
Description} *
\href{https://github.com/wrh2/sputnik/wiki/Phase-1-Test-Plan\#preparation}{Preparation}
*
\href{https://github.com/wrh2/sputnik/wiki/Phase-1-Test-Plan\#test-equipment}{Test
Equipment} *
\href{https://github.com/wrh2/sputnik/wiki/Phase-1-Test-Plan\#test-setup-and-calibration}{Test
setup and calibration} *
\href{https://github.com/wrh2/sputnik/wiki/Phase-1-Test-Plan\#system-tests}{System
Tests} *
\href{https://github.com/wrh2/sputnik/wiki/Phase-1-Test-Plan\#10km-radio-communication-test}{10km
Radio Communication Test} *
\href{https://github.com/wrh2/sputnik/wiki/Phase-1-Test-Plan\#system-controller-test}{System
Controller Test} *
\href{https://github.com/wrh2/sputnik/wiki/Phase-1-Test-Plan\#command-test}{Command
Test}

\subsubsection{Revision History}\label{revision-history}

03/03/16 - Created document (Shan)

03/04/16 - Converted to markdown, proofreading changes (Will)

\subsubsection{Introduction}\label{introduction}

\paragraph{Purpose}\label{purpose}

The purpose of this document is to outline the essential testing that
will be conducted to demonstrate the effectiveness of the Sputnik
Capstone project. This test plan is not intended to be inclusive and
additional testing procedures will be added if deemed necessary by any
of the parties involved in the project.

\paragraph{Testing Procedure}\label{testing-procedure}

All of the testing described in this document will be carried out by one
or more than one member of the Sputnik capstone team. An effort will be
made to have the entire group present for as many of the tests as
possible.

\paragraph{Recordings of Results, witnessing, and
Authorities}\label{recordings-of-results-witnessing-and-authorities}

The results of all testing conducted in this test plan will be displayed
on the project wiki. The tests will be conducted on a pass/fail basis
and any tests that do not pass will be noted in the documentation with
an explanation as to why they did not pass. No authorities or witnesses
outside of the group will be required to be present during testing.

\subsubsection{Reference Documents}\label{reference-documents}

\paragraph{Design Documentation}\label{design-documentation}

\begin{figure}[htbp]
\centering
%\includegraphics{http://i.imgur.com/LNKEclE.png}
\caption{Phase 1 Low Level Diagram}
\end{figure}

The Sputnik Capstone project is composed of two separate modules: The
radio module and the control module. The radio module is home to the
microcontroller with integrated radio transceiver (kwox), while the
control module is designed to eventually house a radiation hardened
watchdog controller that will help reboot the system after debilitating
radiation events. For this project, the radiation-hardened components of
the control module will be replaced with off-the shelf components to
help reduce cost.

\subsubsection{Overview}\label{overview}

\paragraph{Operational Description}\label{operational-description}

The Portland State Aerospace Society is sponsoring this capstone based
on the need for a command, control, and communications system for their
CubeSat project. The focus of this capstone will be rapidly prototyping
the radio module and the control module. Sputnik will eventually be
responsible for long distance communications to and from a 400km low
earth orbit, as well as, controlling and communicating with a payload
that is housed within the CubeSat. On top of fulfilling these duties,
once space bound, it will need to be able to deal with a temperature
range of -40C to 50C and radiation events that could cause components to
latch up.

\subsubsection{Pre-test preparation}\label{pre-test-preparation}

\paragraph{Test equipment}\label{test-equipment}

The equipment needed for the tests is as follows: * Power Supply
sufficient to maintain 1A of current at 3V for approximately 15 minutes
* multimeter (voltmeter) * Oscilloscope * USB to micro-USB cable * Logic
analyzer

\paragraph{Test setup and calibration}\label{test-setup-and-calibration}

The testing setup will be discussed for each case along with any
necessary calibration needed prior to testing.

\subsubsection{System tests}\label{system-tests}

\paragraph{10km Radio Communication
Test}\label{km-radio-communication-test}

The radio is fundamental to the functionality of the Sputnik project. It
provides the communication channel that will link the satellite to the
ground station. Eventually, the radio will need to receive and transmit
data over a distance of approximately 400km; however, for this project,
a transmission distance of 10km is required. The purpose of this test is
to confirm that the radio is capable of 10km transmission and reception.
This test will be performed from one radio board to another and the
testers will verify the distance covered during the test by collection
GPS location data. The test locations will be predetermined based on
both convenience and also where the least restricted signal propagation
path will occur.

\begin{verbatim}
                      |                              |
\end{verbatim}

------------------------- \textbar{} ----------------------------
\textbar{} Test Case Name \textbar{} 10km Radio Communication \textbar{}
Test ID\# \textbar{} 10k\_1.00 \textbar{} Test Writer \textbar{} Shan
Quinney \textbar{} Description \textbar{} The purpose of this test is to
ensure that the radio is capable of transmitting and receiving data at
this distance. \textbar{} Tester Information \textbar{} \textbar{} Name
of Tester \textbar{} \textbar{} Time/Date \textbar{} \textbar{} Hardware
Version \textbar{} Sputnik radio board version 1.00 \textbar{} Setup
\textbar{} Determine location A and location B, where there is a minimal
distance of 10km between points A and B. Have at least one team member
located at location A and at least one other team member located at
location B. Each location will have a Sputnik radio board with
sufficient power supply. Each location will also have a method to verify
GPS and time (cell phone). \textbar{}

\begin{longtable}[c]{@{}lllll@{}}
\toprule
\begin{minipage}[b]{0.06\columnwidth}\raggedright\strut
Step
\strut\end{minipage} &
\begin{minipage}[b]{0.08\columnwidth}\raggedright\strut
Action
\strut\end{minipage} &
\begin{minipage}[b]{0.19\columnwidth}\raggedright\strut
Expected Result
\strut\end{minipage} &
\begin{minipage}[b]{0.12\columnwidth}\raggedright\strut
Pass/Fail
\strut\end{minipage} &
\begin{minipage}[b]{0.11\columnwidth}\raggedright\strut
Comments
\strut\end{minipage}\tabularnewline
\midrule
\endhead
\begin{minipage}[t]{0.06\columnwidth}\raggedright\strut
1
\strut\end{minipage} &
\begin{minipage}[t]{0.08\columnwidth}\raggedright\strut
Radio at location A is made to transmit data.
\strut\end{minipage} &
\begin{minipage}[t]{0.19\columnwidth}\raggedright\strut
Team member at location B confirms receiving data transmitted from
location A.
\strut\end{minipage} &
\begin{minipage}[t]{0.12\columnwidth}\raggedright\strut
\strut\end{minipage} &
\begin{minipage}[t]{0.11\columnwidth}\raggedright\strut
\strut\end{minipage}\tabularnewline
\begin{minipage}[t]{0.06\columnwidth}\raggedright\strut
2
\strut\end{minipage} &
\begin{minipage}[t]{0.08\columnwidth}\raggedright\strut
Radio at location B sends confirmation signal.
\strut\end{minipage} &
\begin{minipage}[t]{0.19\columnwidth}\raggedright\strut
Team member at location A confirms receiving data from radio at location
B.
\strut\end{minipage} &
\begin{minipage}[t]{0.12\columnwidth}\raggedright\strut
\strut\end{minipage} &
\begin{minipage}[t]{0.11\columnwidth}\raggedright\strut
\strut\end{minipage}\tabularnewline
\bottomrule
\end{longtable}

\textbf{Overall Test Result:}

\paragraph{System controller Test}\label{system-controller-test}

The system controller is the guardian of the system. It is present to
ensure that the system is functioning correctly and that if any
unintended event causes component latch-up or system errors, the system
can be cycled or rebooted to return stability. This control system is
the other half of the project. Eventually, this system will consist of a
radiation hardened microcontroller (ATMegaS128) with supporting
radiation hardened LDO. For the purpose of prototyping, the controller
is a standard, off-the-shelf ATMega128 chip.

To test the control system, a method to simulate a latch-up event will
be used to trigger the watchdog into action. Outlined is the kwox
lock-up test. In this test, the crystal on the kwox will be shorted to
cause an error in the radio system. The ATMega should sense that the
radio is no longer functioning properly and trigger the reset line on
the kwox to initiate a reboot.

\begin{verbatim}
                      |                              |
\end{verbatim}

------------------------- \textbar{} ----------------------------
\textbar{} Test Case Name \textbar{} System Controller Test \textbar{}
Test ID\# \textbar{} ATM\_1.00 \textbar{} Test Writer \textbar{} Shan
Quinney \textbar{} Description \textbar{} The purpose of this test is to
demonstrate the effectiveness of the watchdog to restart key system
functionality after radiation events. \textbar{} Tester Information
\textbar{} \textbar{} Name of Tester \textbar{} \textbar{} Time/Date
\textbar{} \textbar{} Hardware Version \textbar{} Board Rev.1, Filter
Rev.1, Wire antenna \textbar{} Setup \textbar{} \textbar{}

\begin{longtable}[c]{@{}lllll@{}}
\toprule
\begin{minipage}[b]{0.06\columnwidth}\raggedright\strut
Step
\strut\end{minipage} &
\begin{minipage}[b]{0.08\columnwidth}\raggedright\strut
Action
\strut\end{minipage} &
\begin{minipage}[b]{0.19\columnwidth}\raggedright\strut
Expected Result
\strut\end{minipage} &
\begin{minipage}[b]{0.12\columnwidth}\raggedright\strut
Pass/Fail
\strut\end{minipage} &
\begin{minipage}[b]{0.11\columnwidth}\raggedright\strut
Comments
\strut\end{minipage}\tabularnewline
\midrule
\endhead
\begin{minipage}[t]{0.06\columnwidth}\raggedright\strut
1
\strut\end{minipage} &
\begin{minipage}[t]{0.08\columnwidth}\raggedright\strut
Use a metal tool to cause a short across the crystal
\strut\end{minipage} &
\begin{minipage}[t]{0.19\columnwidth}\raggedright\strut
The kwox will loose the signal from the crystal.
\strut\end{minipage} &
\begin{minipage}[t]{0.12\columnwidth}\raggedright\strut
\strut\end{minipage} &
\begin{minipage}[t]{0.11\columnwidth}\raggedright\strut
\strut\end{minipage}\tabularnewline
\begin{minipage}[t]{0.06\columnwidth}\raggedright\strut
2
\strut\end{minipage} &
\begin{minipage}[t]{0.08\columnwidth}\raggedright\strut
Probe the UART line between the controller and the kwox to determine
that the life line signal is lost
\strut\end{minipage} &
\begin{minipage}[t]{0.19\columnwidth}\raggedright\strut
The UART line will be free of any signal between the kwox and the
controller.
\strut\end{minipage} &
\begin{minipage}[t]{0.12\columnwidth}\raggedright\strut
\strut\end{minipage} &
\begin{minipage}[t]{0.11\columnwidth}\raggedright\strut
\strut\end{minipage}\tabularnewline
\begin{minipage}[t]{0.06\columnwidth}\raggedright\strut
3
\strut\end{minipage} &
\begin{minipage}[t]{0.08\columnwidth}\raggedright\strut
Monitor the controller to ensure that the reset line on the kwox has
been activated
\strut\end{minipage} &
\begin{minipage}[t]{0.19\columnwidth}\raggedright\strut
The reset line on the kwox will be activated in an effort to reboot the
device.
\strut\end{minipage} &
\begin{minipage}[t]{0.12\columnwidth}\raggedright\strut
\strut\end{minipage} &
\begin{minipage}[t]{0.11\columnwidth}\raggedright\strut
\strut\end{minipage}\tabularnewline
\bottomrule
\end{longtable}

\textbf{Overall Test Result:}

\paragraph{Command test}\label{command-test}

Send command to module, blink an LED or toggle GPIO pin

\begin{verbatim}
                      |                              |
\end{verbatim}

------------------------- \textbar{} ----------------------------
\textbar{} Test Case Name \textbar{} Command Test \textbar{} Test ID\#
\textbar{} CMD\_1.00 \textbar{} Test Writer \textbar{} Will Harrington
\textbar{} Description \textbar{} The purpose of this test is to
demonstrate the effectiveness of the system controller to execute
commands \textbar{} Tester Information \textbar{} \textbar{} Name of
Tester \textbar{} \textbar{} Time/Date \textbar{} \textbar{} Hardware
Version \textbar{} \textbar{} Setup \textbar{} \textbar{}

\begin{longtable}[c]{@{}lllll@{}}
\toprule
Step & Action & Expected Result & Pass/Fail & Comments\tabularnewline
\midrule
\endhead
1 & Send command & Successful send & &\tabularnewline
2 & Observe LED on prototype & LED lights up & &\tabularnewline
\bottomrule
\end{longtable}

\textbf{Overall Test Result:}

\end{document}
